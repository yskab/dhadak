भारत में COVID-19 की लहर के दौरान, मैंने देखा, बहुत लोग घर पर ऑक्सीजन के स्तर की निगरानी के लिए मेडिकल स्टोर और ऑनलाइन ऑक्सीमीटर खरीद रहें थे। मैं काफ़ी उत्सुक था, यह जानने के लिये कि उंगली से जुड़ा यह छोटा उपकरण हृदय गति और ऑक्सीजन संतृप्ति की गणना कैसे कर सक़ता है? इस सवाल ने मुझे शोध करने और समझने को प्रेरित किया कि वास्तव में ऑक्सीमीटर कैसे काम करता है और क्या मैं इसे खुद पुरा बना सक़ता हूं। इस प्रक्रिया में मुझे एहसास हुआ, सभी विवरणों को एक समझ दस्तावेज़ में लिखना सही होगा जो मेरे लिए एक स्मृति के रूप में काम करेगा और दूसरों के लिए भी मददगार हो सक़ता है।
साथ ही, मैं \LaTeX \hspace{1pt} सीखना शुरू करना चाहता था ताकि भविष्य की सभी परियोजनाओं के लिए सुंदर लिखित दस्तावेज़ तैयार किये जा सके।\bigskip

इस लेखन मे, ऑक्सीमीटर क्या है, ऑक्सीजन संतृप्ति और हृदय गति प्राप्त करने के लिए इलेक्ट्रॉनिक सर्किट का उपयोग कैसे किया जा सक़ता है, इसका एक संक्षिप्त सिद्धांत प्रस्तुत करने का प्रयास किया गया है। आवश्यक सर्किट तत्वों और ऐल्गोरिद्म पर भी चर्चा की गई है। यह माना गया है कि पाठक को निम्नलिखित विषयों का ज्ञान हैै:

\begin{itemize}
	\item बुनियादी सर्किट विश्लेषण \& इलेक्ट्रॉनिक तत्व
	\item ऑप एंप \& ट्रांजिस्टर के नियम
\end{itemize}

\bigskip

मैं इलेक्ट्रॉनिक्स सीखने कि यात्रा में नौसिखिया हूं, इसलिए इस लेख में गलतियां हो सकती हैं। प्रस्तुत जानकारी की सत्यता की जांच मेरे अलावा किसी अन्य व्यक्ति ने नहीं की है।
यह लेख इसलिए लिखा गया था ताकि ज्ञान का स्रोत बन सके कि ऑक्सीमीटर कैसे बनाया जाता है या ऑक्सीमीटर कैसे काम करता है। प्रस्तुत डिज़ाइन का उपयोग किसी भी जीवन समर्थन प्रणाली में नहीं किया जाना है।

\bigskip
यह दस्तावेज़ TexStudio editor मे लिखा गया है।	

सभी चित्र Inkscape से तैयार किए गए है।

लेखाचित्र GNU Octave मे बनाये गये है।

PCB \& Schematics को Kicad मे डिज़ाइन किया गया है।

एन्क्लोज़र FreeCAD मे डिज़ाइन किया गया है।

\medskip
परियोजना से संबंधित सभी फाइलें यहां है: \url{https://github.com/yskab/dhadak}

वीडियो श्रंखला: youtube.com/

\medskip
यदि कोई गलतियां मिलती है या सुधार करना चाहते हैं तो कृपया Github पर संपर्क करें। 

\medskip
पोस्ट और संदेशों के लिए Twitter पर मुझ तक पहुंच सकते हैं 
\href{https://twitter.com/yskabhijeet}{@yskabhijeet}

\includegraphics[width=0.2\textwidth]{../common/cc.png}\par

\vfill
\hfill
{\Large\bfseries वाई एस के अभिजीत}\par
\hfill
\large \today