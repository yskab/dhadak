During the COVID-19 wave in India, I saw a lot of people buying oximeters from medical stores and online to monitor oxygen levels at home. I was quite intrigued, how can a small device plugged to a finger could calculate heart rate and oxygen saturation? This question got me started to research and understand how these things actually work and whether I can make one from scratch. In the process I realized, it would be helpful to write all details in an understanding document which will serve as a memory to me and can be of help to others too.
Also, I wanted to start learning \LaTeX \hspace{1pt} so as to create beautiful written documents for all future projects.\medskip

This writing tries to present a brief theory of what oximetery is, how electronic components can be used to get the oxygen saturation value and heart rate. Required circuit block elements and algorithms have also been discussed. It has been assumed that reader has knowledge of:

\begin{itemize}
	\item Basic circuit analysis \& electronic components
	\item Opamp \& transistor working
\end{itemize}

\bigskip

I am a beginner in my electronics journey, hence there can be mistakes in this article. The accuracy of information presented has not been inspected by any other person than myself.
This guide was written so that it can become a source of knowledge to anyone trying to understand how to make an oximeter or how oximeter works. Presented design is not be used in any life support systems.

\bigskip
This document was written in TexStudio editor.

All pictures have been produced with Inkscape.

Plots were generated in GNU Octave.

PCB \& Schematics designed using Kicad.

Enclosure was designed in FreeCAD.

\medskip
All project related files are at: \url{https://github.com/yskab/dhadak}

Video series: youtube.com/

For reporting any mistakes or corrections raise an issue on Github.

You can reach me on Twitter for posts and messages \href{https://twitter.com/yskabhijeet}{@yskabhijeet}

\includegraphics[width=0.2\textwidth]{../common/cc.png}\par

\vfill
\hfill
{\Large\bfseries Y S K Abhijeet}\par
\hfill
\large \today

